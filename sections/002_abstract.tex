本論文では現実空間を移動しながら遠くにいる友人と共に遠隔地の風景での行動を可能にするテレイグジスタンス 手法を提案し,その手法を基に実装した遠隔旅行システムについて述べる.

自身を遠隔地に存在しているかのようにするテレイグジスタンスという技術が存在する. この技術を利用し工場での作業やスポーツ観戦,旅行などを遠隔地から行う研究が行われている. これらの研究の多くは人型のロボットや完全没入型 VR を利用しており, 高い没入感での体験が可能であるが, ユーザの周囲の状況が判断できないために自由な移動ができない. 

これらの問題を解決するために透過型 MR ヘッドマウントディスプレイを用いてユーザをアバターとして遠くにいる友人の周りに表示し,さらに遠隔地の風景と現実の風景を重畳し,遠隔地の道路部分と現実の道路以外の部分を切り抜く手法を提案する. 

提案手法を基に遠隔旅行をするシステムを実装した.
システムにより複数のユーザ達は自分以外のアバターと共にあたかも遠隔地での旅行体験が可能となる. ユーザのアバターはヘッドマウントディスプレイの位置と同期しており, ユーザが移動を行うとアバターも同様に移動を行う. 遠隔地の風景表示は全天球パノラマ画像を用いてユーザ周囲の風景に重畳する. その際に視野確保のため道路部分を切り取る. これにより現実の道路を視認可能になり他者との衝突などの危険を抑制し現実空間での移動を可能にする.

本システムを評価するため学生10名に対し実験を行なった.
実験は本システムと関連したシステムのHoloTourを被験者に使用してもらい,その後テレイグジスタンスシステムの有用性の評価に使用される Presence Questionnaire と個人の没入感の感じ方を調べるのに使用される Immersive Tendencies Questionnaire という2つのアンケートに回答してもらった.
結果,本システムと HoloTour に有意差は見られなかったが,没入感を感じにくいほど本システムを使用した時遠隔地をよりリアルに感じることがわかった. 