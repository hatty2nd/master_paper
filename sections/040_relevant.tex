\section{関連研究}
本研究を行うにあたり5つの点に着目し,関連する研究を紹介する. 1つ目は遠隔地にいる他ユーザを表示する点, 2つ目はバーチャルヒューマンを利用する点, 3つ目は風景を変化させる点, 4つ目は移動を伴う環境で使用する点, 5つ目はMRを利用している点である.以下紹介した研究と本研究との関係を述べる.

% + MR,移動を伴うシステム,バーチャルヒューマン

\subsection{他ユーザの表示をする研究}
ユーザの存在を遠隔地に表示させるためにユーザの3Dモデルを用いた手法が存在する.

濵上らは対話相手に自身の空間へと侵入されている感覚「被侵入感」を提示し,テレプレゼンスの向上を目指すビデオチャットシステム「ドアコム AR」を開発した\cite{doorcom}.
ドアコムARはKinectにより取得したユーザのカラー情報,深度情報から3次元点群データを取得する.
取得した点群データを遠隔地にいる別のユーザの前に表示することで同室感のあるビデオチャットが可能になる.
このようにユーザの姿を遠隔地に表示するシステムは他にも存在する.
CutlerらはHololens\cite{Hololens}を利用し遠隔地にいる友人の存在をリアルタイムで表示するシステムHolopotationを開発した\cite{holopotation}.
Holopotationは深度情報を取得可能なカメラを複数使いユーザを撮影し、ユーザの3Dモデルを作成する.
そして作成した3Dモデルを遠隔地にいる別のユーザの前にHololensを通して表示することで,遠隔地にいるユーザとコミュニケーションを取ることができる.

\clearpage

\subsection{バーチャルヒューマンを用いた研究}
本手法は他ユーザをアバターとして表示する.このようにコンピュータによって表示される人間はバーチャルヒューマンと呼ばれ,人の認知に与える影響が調査されている.

Devidらはユーザーの心理的苦痛に関連する情報を共有する,魅力的な対面式対話を作成するように設計された SimSensei Kiosk というバーチャルヒューマンのインタビュアーを開発した\cite{kiosk}.SimSensei Kioskはうつ病や不安といった心理的苦痛を自動評価することで,会話相手であるユーザに対し快適な会話や情報共有を可能とする.

デジタルエミリーというリアリティのあるバーチャルヒューマンが存在する\cite{digitalemily}.しかし人はこのようなリアリティのあるバーチャルヒューマンに対して不気味だと感じることがある.Maryamらはこうしたリアルなバーチャルヒューマンを見たときの人間の脳波を学習し,不気味と思うバーチャルヒューマンを調査した\cite{maryam}.

\clearpage

\subsection{風景を変化させる研究}
パノラマ画像を球体オブジェクトに貼り付け風景を変化させる手法はChenらによって提案され\cite{shenchang},その後この手法を用いたVR研究が多く行われるようになった.

RuygらはヘッドマントディスプレイOculus Riftを用いて Google Street View を表示する Oculus Google Street View を実装した\cite{oculus}.これによりユーザは Google Street View が存在する場所を表示し,あたかもその場所に行ったかのような体験が可能となる.

Calagariらはスポーツ放送のカメラ映像から広角パノラマ画像を生成することで没入型のスポーツ観戦システムを実装した.実装したシステムについて評価した結果ほとんどの評価参加者が没入感を感じた\cite{sprots}.

\clearpage

\subsection{移動を伴う環境で使用する研究}
移動を伴う環境で使用する研究として
Nguyen らはユーザがジョギングする際に過去の自分のスピードを元に生成された影を AR により走らせることで,ジョギングのモチベーション維持を図るシステム Fitnamo を提案している\cite{fitnamo}. このシステムは過去のユーザのジョギングデータを元に,ユーザ自身の平均速度, または最高速度で走る影をGoogleGlass に投影し,競争心を持たせることでモチベーションの動機付けを図っている. 
%図 3 はこの システムの使用例である. 赤く表示されているのは 今までのユーザの平均速度で走った場合のユーザの 位置を示している.

ナイアンティックと株式会社ポケモンはスマートフォン向け位置情報ゲームアプリケーションのポケモンGOを開発した\cite{pokego}.
ポケモンGOはスマートフォンのGPS機能を使用しながら移動することでキャラクターの捕獲・育成・交換・バトルを画面上で行うことができる。

\clearpage

\subsection{MRを用いた研究}
MRを用いた研究としてMichael らは複数人でチームを組み現実世界に投影した仮想ドローンを撃ち落とすMRマルチプレイヤーゲームを提案している\cite{mrgame}.Michaelらのシステムは屋外ゲームにおいて,プレイヤー間の相対的な位置を追跡することで複数人でのプレイを可能にした.相対的な位置の推定はGPSを用いて行っている. 

%差
本システムはパノラマ画像により周囲の風景全体を変化させることでよりMRとしての没入感の向上を図る.

HaleyらはHololensにより,ペルーのマチュピチュ遺跡やイタリアのナヴォーナ広場の環境音を流し,周囲の映像を表示することであたかもその場所に言ったかのような体験を可能にするHoloTourを開発した\cite{holotour}.環境音は指向性マイクを利用し録音し周囲の風景は$360^\circ$カメラを使い撮影した.

\clearpage

\subsection{関連研究との差異}
濵上ら,Cutlerらのシステムはリアルなユーザのモデルを表示可能であるが,自分がKinectやカメラに映る範囲にいなければならないため,移動範囲に制限が存在する.
本手法はユーザの存在をアバターとして表示し,その制御をHololensの自己位置推定機能により行なっているため,設置が必要な機器がなく移動範囲に制限を無くした.

DevidらやMaryamらはバーチャルヒューマンの研究を行なっている.本手法ではMRを用いてバーチャルヒューマンを表示することにより,ユーザが移動しながら自身の存在を遠隔地に表示する.

Ruygら,Calagariらのシステムは仮想空間の移動にコントローラを用いるか,移動せずその場で周りを見渡すことで仮想空間を体験する.これに対し本手法はMRを用いているためユーザは実際に歩きながら仮想空間の風景の中で行動ができる.これにより身体的動作を伴った体験になるためユーザはより現実感のある体験が可能になる.


NguyenらやMichaelら,ナイアンティックが開発したシステムはARやMRを用いて現実世界にオブジェクトを投影している.本手法では現実世界に対し遠隔地の風景を重畳することで,ユーザの周囲ほとんどが仮想の風景へと変化する.

Haleyらのシステムは高性能なカメラと指向性マイクにより没入感の高い体験が可能であるが,現実環境を視認できず移動範囲も制限がある.本システムは表示する遠隔地の風景から道路部分を切り抜くことで,現実の道路を視認可能にしている.そのため実際に移動しながら遠隔地に行ったかのような体験を可能にしている.

%\subsection{アウトドアレクリエーションの拡張}
%アウトドアレクリエーションを拡張するシステムとして屋外ゲーム,屋外ナビゲーションシステムについて述べる.
%屋外ナビゲーション
%Aysegul Dogangunらは身体活動を改善するための個別の推奨事項を生成するモバイルアプリケーションを開発した.[\cite{Aysegul}Aysegul Dogangun 2017]. このアプリケーションはユーザーが必要なすべての活動データを収集する.収集したデータに基づいて,アプリケーションは毎日のルーチンで不足した運動量を補完するために,毎日のルーチンよりも運動量が増加した活動を自動的に推奨する.
%Maaretostiらは,非自発的なハイキングシステムであるHOBBITを開発した[\cite{maaret} Maaret Posti 2014].このシステムはユーザが興味のある領域のハイキングコースを提案することに加え,ユーザ周辺のWi-Fi機器を検出し通知することで,ユーザが一人で歩くことを支援している.
%Aysegul DogangunらやMaaret Postiらはモバイル端末への活動量の通知や提案コースと周辺のユーザを通知する事でレクリエーションの拡張を図っている.
%本システムではMRを用いた視覚的情報の操作によるレクリエーションの拡張を図る.

%本システムと同様にMRを用いた手法としてMichael Bonfertらは複数人でチームを組み現実世界に投影した仮想ドローンを撃ち落とすMRマルチプレイヤーゲームを提案している[\cite{Michael}Michael Bonfert 2017].Michael Bonfertらのシステムは屋外ゲームにおいて,プレイヤー間の相対的な位置を追跡することで複数人でのプレイを可能にした.相対的な位置の推定はGPSを用いて行っている. 本システムはパノラマ画像により周囲の風景全体を変化させることでよりMRとしての没入感の向上を図る.
%位置付け
%屋外ゲーム
%Aysegulらは身体活動を改善するための個別の推奨事項を生成するモバイルアプリケーションを開発した\cite{Aysegul}. このアプリケーションはユーザーが必要なすべての活動データを収集する.収集したデータに基づいて,アプリケーションは毎日のルーチンで不足した運動量を補完するために,毎日のルーチンよりも運動量が増加した活動を自動的に推奨する.
%Maaretostiらは,非自発的なハイキングシステムであるHOBBITを開発した\cite{maaret}.このシステムはユーザが興味のある領域のハイキングコースを提案することに加え,ユーザ周辺のWi-Fi機器を検出し通知することで,ユーザが一人で歩くことを支援している.
%Aysegul DogangunらやMaaret Postiらはモバイル端末への活動量の通知や提案コースと周辺のユーザを通知する事でレクリエーションの拡張を図っている.
%本システムではMRを用いた視覚的情報の操作によるレクリエーションの拡張を図る.

%Devid Debultらはユーザーの心理的苦痛に関連する情報を共有する,魅力的な対面式対話を作成するように設計された Sim Sensei Kiosk というバーチャルヒューマンのインタビュアーを開発した.
%SimSensei Kioskはうつ病や不安といった心理的苦痛を自動評価することで,会話相手であるユーザに対し快適な会話や情報共有を可能とする.

%Maryam Mustafaらはバーチャルヒューマンが人の認知に与える影響を調査している.人はデジタルエミリーのようなリアリティのあるバーチャルヒューマンに対して不気味だと感じることがある.Maryam Mustafaらはこうしたリアルなバーチャルヒューマンを見たときの人間の脳波を学習し,不気味と思うバーチャルヒューマンを調査した.
%本システムでは,MRを用いてバーチャルヒューマンを表示することにより,ユーザがレクリエーションを行う際に複数人でのレクリエーション体験の演出を図る.
