\subsection{データ取得部}
データ取得部は,出力部でのユーザ識別,ユーザ毎の風景表示のため,ユーザ情報,遠隔地のStreetViewに関する情報を取得する処理群である.データは遠隔地設定ページでユーザが入力した値を使用する.取得したデータはデータベースに保存し管理する.本システムではUnityの標準クラスで利用可能な軽量データ交換フォーマットJSONを利用しデータを管理する\cite{json}.
以下にJSONに含まれる各データの詳細を示す.

\begin{table}[ht]
  \begin{tabular}{| c | l |}
  \hline
    ユーザID & ユーザを識別するための固有ID \\ \hline
    表示遠隔地 & 遠隔地の情報からユーザが現在見ている遠隔地の場所を抽出するための値\\ \hline
    遠隔地の情報群 & 指定遠隔地の経度,緯度,パノラマID \\ \hline
  \end{tabular}
\end{table}

\clearpage

\subsubsection{ユーザ情報取得}
ユーザ情報取得部ではユーザの識別のために使用する固有のユーザIDを取得する.ユーザIDとしてユーザが使用するHololensの端末IDを使用する. 遠隔旅行をするコースの設定用Webページでユーザが入力したHololensの端末IDをサーバに送信し,データベースを更新する.これによりシステムを使用する際にどのユーザがどのアバターの制御するかを識別する.

\clearpage

\subsubsection{StreetView情報取得部}
コース設定部ではユーザが選択した 2つのポイ ントを出発地と目的地とし, 徒歩での移動コースを 生成する. コースの生成は GoogleMapsApi を用いて行なう\cite{googlemapsapi}. 
生成したコース上にStreetViewを取得する座標を一定の間隔で設定する.Google が定めた StreetView 認定要件において StreetView 間の距離は 3 メートルとなっていたため\cite{svrule}. コース上に設定するStreetViewを取得する座標の間隔は 3 メートルとした. StreetViewを取得する座標の設定方法はまずコース上からカーブや曲がり角を全て抽出し, それぞれの座標を記録する. その後抽出した座標間を 3 メートル間隔で分割し, StreetViewを取得する座標を設定する. 
その後Google Maps APIを用いて3メートル感覚で取得した座標からStreetViewのパノラマ画像が持つ固有のパノラマIDを取得する.取得したパノラマIDと座標をデータベースに書き込み更新をする.データベースへの書き込みは出発地点の座標からはじめ,同じパノラマIDを持つ座標は書き込みを行わないようにした.これによりデータベースでパノラマIDが重複する事を防ぐ.



%図\ref{figure:setcource}は

%図 9 の例ではコース上から 3 つの曲がり角の座標を記録し, 更に各座標間において 3 メートル間隔で StreetView を取得するポイントを計 78 ポイント設定している.


