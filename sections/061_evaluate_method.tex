\subsection{評価手法}
実験は被験者10人に対して行なった.被験者は本システムと関連研究であげたHoloTourを使用して遠隔地の風景を体験した後アンケートに回答した.
この時システムを使用する順番はランダムに決定し,体験する遠隔地は2つシステムで同じ場所に統一する.
実験で体験する遠隔地はHoloTourで体験可能なイタリアのナヴァーナ広場とした.
本システムを使用する際被験者の前にアバターを表示し, 我々がノートパソコンを通してアバターを操作した.
%HoloTourで体験できる風景は現時点でペルーの世界遺産マチュピチュとイタリアのナヴァーナ広場のみであったため,同じ風景を
システム使用後に回答するアンケートはWitmerとSingerにより提案され,その後UQOサイバー心理学研究室により改訂されたPresence Questionnaire(PQ)と, Immersive Tendencies Questionnaire(ITQ)を利用し2つのシステムの比較を行う\cite{pq}\cite{itq}.
PQはプレゼンスシステムの優位性を測る質問であり,以下の項目を評価することでシステムの優位性を算出できる.

\begin{table}[ht]
 \begin{center}
  \begin{tabular}{| l |}
  \hline
 現実感高さ(現実感) \\ \hline
システムで行える行動のレベル(行動) \\ \hline
 表示や操作などの質(インタフェース) \\ \hline
 仮想環境でどれだけ探索や調査ができたか(調査)\\ \hline
 仮想環境での自身のパフォーマンスの熟練度(パフォーマンス)\\ \hline
 音響の質(音響)\\ \hline
 触覚フィードバックの質(触覚)\\ \hline
  \end{tabular}
  \end{center}
\end{table}
ITQは個人が感じる没入感の傾向を測る質問であり,スコアが高いほど没入感を感じやすいことを示している.
PQの結果について2標本$t$検定を行い比較を行うシステムの平均値に有意差が存在するか調査する.
この時有意水準を$0.05$とした.
またPQのスコアとITQのスコアの関係を調査し本システムの評価を行う.

比較は PQ から本システムに実装されていない音響と触覚に関する項目を抜いた現実感,行動,インタフェース,調査,自身のパフォーマンスの5項目について行う.
PQの質問は合計19問で,それぞれ7段階で評価してもらった.
スコアは原著論文の方法通り現実感,行動,調査,自身のパフォーマンスはスコアが高い方が優れている事とし,逆にインタフェースはスコアが低い方が優れている事とする.

\clearpage