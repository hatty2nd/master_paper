%7 成果
\section{成果}
%時間の錯覚効果を利用し, 就寝時刻を理想の時刻に近づけることが可能となった.
%しかしながら, 自然に就寝に誘うという目標の達成は不十分であった.
本論文では移動を伴う環境下において遠くにいる人とともに遠隔地の風景での行動を可能にするテレイグジスタンスの手法を提案し,実装したシステムにより評価を行った.
その結果移動など身体動作を伴ったテレイグジスタンス システムは,ITQのスコアが低く没入感を感じにくいユーザに対して有効であることがわかった.