\documentclass[12pt]{jarticle}
\usepackage[dvipdfmx]{graphicx}

% 文献番号や目次がハイパーリンク化する
\usepackage[dvipdfmx]{hyperref}
% 表の幅指定
\usepackage{multirow}
% 余白の設定
\usepackage[top=30truemm,bottom=30truemm,left=25truemm,right=25truemm]{geometry}

\fboxsep=0pt
\fboxrule=1pt

\makeatletter
\def\thefigure{\thesection.\arabic{figure}}
\@addtoreset{figure}{section}

\renewcommand\section{\@startsection {section}{1}%section周りのスペース
{\z@}%
{1.5\Cvs \@plus.5\Cdp \@minus.2\Cdp}%
{.5\Cvs \@plus.3\Cdp}%
{\reset@font\LARGE\bfseries}}%文字サイズフォント

\renewcommand\subsection{\@startsection {subsection}{2}%subsection周りのスペース
{\z@}%
{1.5\Cvs \@plus.5\Cdp \@minus.2\Cdp}%
{.5\Cvs \@plus.3\Cdp}%
{\reset@font\Large\bfseries}}%文字サイズフォント

\renewcommand\subsubsection{\@startsection {subsubsection}{3}%subsubsection周りのスペース
{\z@}%
{1.5\Cvs \@plus.5\Cdp \@minus.2\Cdp}%
{.5\Cvs \@plus.3\Cdp}%
{\reset@font\large\bfseries}}%文字サイズフォント
\makeatother

\begin{document}

\tableofcontents

\clearpage

\section{管理アプリケーションに於ける画面遷移図}
管理アプリケーションに於ける画面遷移図を図\ref{fig:transition_control}に示す.

\begin{figure}[ht]
	\begin{center}
		\centering
		\fbox{\includegraphics[width=0.9\columnwidth]{images/spec_external/transition_control.png}}
		\caption{画面遷移図}
		\label{fig:transition_control}
	\end{center}
\end{figure}

\clearpage

\section{管理アプリケーションの各画面の説明}
管理アプリケーションの各操作画面について, 部位の説明を行う.

\subsection{待機画面}
待機画面は, 管理アプリケーションの起動直後に表示される, 現在の状態を表示して操作待ちを行う画面である.

待機画面の画像を図\ref{fig:annotation_control_main}に示す.

\begin{figure}[ht]
	\centering
	\begin{center}
		\centering
		\fbox{\includegraphics[width=0.45\columnwidth]{images/spec_external/annotation_control_main.png}}
		\caption{待機画面}
		\label{fig:annotation_control_main}
	\end{center}
\end{figure}

\begin{enumerate}
	\item 時刻表示 \\
		算出された時刻を表示する.
		未だ設定されていないなら, 現在時刻を表示する.
	\item 表示デバイス連携ボタン \\
		タップされると, 表示デバイス検索画面へ移動する.
	\item サブメニュー表示 \\ 
		タップされると, サブメニュー (図 \ref{fig:screenshot_control_main_menu}) を表示する. \\
		サブメニューの「秒針表示を切り替え」がタップされると, 上記の時刻表示に際し, 秒の部分を追加で表示する (図 \ref{fig:screenshot_control_main_second}) .
	\item 設定ボタン \\ 
		タップされると, 設定画面へ移動する.
\end{enumerate}

\begin{figure}[ht]
	\centering
	\begin{tabular}{p{0.45\columnwidth}p{0.05\columnwidth}p{0.45\columnwidth}}
		\centering
		\fbox{\includegraphics[width=0.45\columnwidth]{images/spec_external/screenshot_control_main_menu.png}}
		\caption{待機画面 (サブメニューを表示)}
		\label{fig:screenshot_control_main_menu}
		&&
		\centering
		\fbox{\includegraphics[width=0.45\columnwidth]{images/spec_external/screenshot_control_main_second.png}}
		\caption{待機画面 (秒部を表示)}
		\label{fig:screenshot_control_main_second}
	\end{tabular}
\end{figure}

\clearpage

\subsection{設定画面}
設定画面は, 時刻変化を設定する画面である. 生活リズム推定部からのデータ受信も行う.

設定画面の画像を図\ref{fig:annotation_control_assign}に示す.

\begin{figure}[ht]
	\centering
	\begin{center}
		\centering
		\fbox{\includegraphics[width=0.45\columnwidth]{images/spec_external/annotation_control_assign.png}}
		\caption{設定画面}
		\label{fig:annotation_control_assign}
	\end{center}
\end{figure}

\begin{enumerate}
	\item 「理想の就寝時刻」の設定 \\
		タップされると, 「理想の就寝時刻」を設定する為のダイアログ (図 \ref{fig:screenshot_control_assign_dial}) を開く.
	\item 「いつもの就寝時刻」の設定 \\
		タップされると, 「いつもの就寝時刻」を設定する為のダイアログ (図 \ref{fig:screenshot_control_assign_dial}) を開く.
	\item 「変化曲線」の設定 \\
		タップされると, 「変化曲線」を選択する為のドロップダウンリスト (図 \ref{fig:screenshot_control_assign_drop}) を展開する.
	\item 選択された「変化曲線」 \\
		選択された変化曲線のイメージを表示する.
	\item 「いつもの就寝時刻」提案ボタン \\ 
		タップされると, ユーザの普段の生活から推定された「いつもの就寝時刻」を Firebase から取得し, 設定に反映する.
	\item 設定リセットボタン \\
		タップされると, 全ての設定をリセットする.
	\item 設定確定ボタン \\
		タップされると, 設定を確定し, 待機画面へ移動する.
	\item 戻る \\
		タップされると, 設定を中断し, 待機画面へ移動する.
\end{enumerate}

\begin{figure}[ht]
	\begin{tabular}{p{0.45\columnwidth}p{0.05\columnwidth}p{0.45\columnwidth}}
		\centering
		\fbox{\includegraphics[width=0.45\columnwidth]{images/spec_external/screenshot_control_assign_dial.png}}
		\caption{時刻設定ダイアログ}
		\label{fig:screenshot_control_assign_dial}
		&&
		\centering
		\fbox{\includegraphics[width=0.45\columnwidth]{images/spec_external/screenshot_control_assign_drop.png}}
		\caption{変化曲線ドロップダウン}
		\label{fig:screenshot_control_assign_drop}
	\end{tabular}
\end{figure}

\clearpage

\subsection{表示デバイス検索画面}
表示デバイス検索画面は, 接続待機中の表示デバイスを検索し, 接続を開始する操作を行う画面である.

表示デバイス検索画面の画像を図\ref{fig:annotation_control_scan}に示す.

\begin{figure}[ht]
	\begin{tabular}{p{0.45\columnwidth}p{0.05\columnwidth}p{0.45\columnwidth}}
		\centering
		\fbox{\includegraphics[width=0.45\columnwidth]{images/spec_external/annotation_control_scan.png}}
		\caption{表示デバイス検索画面}
		\label{fig:annotation_control_scan}
		&&
		\centering
		\fbox{\includegraphics[width=0.45\columnwidth]{images/spec_external/annotation_control_scan_found.png}}
		\caption{表示デバイス検索画面 (デバイスを発見後)}
		\label{fig:annotation_control_scan_found}
	\end{tabular}
\end{figure}

\begin{enumerate}
	\item デバイス検索開始ボタン \\
		タップされると, 現在接続可能な \mbox{Bluetooth LE} モジュールの検索を開始する. \\
		検索開始から 6 秒後に自動的に検索は終了する.
	\item 接続先候補一覧 \\
		検索で発見された \mbox{Bluetooth LE} モジュールを一覧表示する. \\
		タップされると, Bluetooth 接続を開始し, 待機画面へ移動する.
	\item デバイス検索中断ボタン \\
		タップされると, \mbox{Bluetooth LE} モジュールの検索を中断する.
	\item 戻る \\
		タップされると, 接続の確立を中断し, 待機画面へ移動する.
\end{enumerate}

\clearpage

\section{状態推定アプリケーションの画面構成}
\mbox{Android Wear}上で動作する, 状態推定アプリケーションの画面構成を図\ref{fig:annotation_estimate}に示す.

\begin{figure}[ht]
	\begin{center}
		\centering
		\fbox{\includegraphics[width=0.5\columnwidth]{images/spec_external/annotation_estimate.png}}
		\caption{状態推定アプリケーション}
		\label{fig:annotation_estimate}
	\end{center}
\end{figure}

\begin{enumerate}
	\item 推定開始ボタン \\
		タップされると, 状態推定サービスを起動する. \\
		IllusionClock 利用中, サービスは常に起動状態を保っておく.
	\item 推定停止ボタン \\
		タップされると, 状態推定サービスを停止する.
\end{enumerate}

\clearpage

\section{表示デバイスの構成}
表示デバイスの物理的構成を図\ref{fig:annotation_display}に示す.

\begin{figure}[ht]
	\begin{center}
		\centering
		\fbox{\includegraphics[width=0.8\columnwidth]{images/spec_external/annotation_display.png}}
		\caption{表示デバイス}
		\label{fig:annotation_display}
	\end{center}
\end{figure}

\begin{enumerate}
	\item 表示ディスプレイ \\
		算出された時刻を表示する.
	\item 接続状態表示ランプ \\
		接続が確立されているなら緑色に, そうでないなら赤色に点灯する.
	\item 照度センサ \\
		環境の照度を計測している.
\end{enumerate}

\end{document}
